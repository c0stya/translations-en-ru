\documentclass{article}

\usepackage{PRIMEarxiv}
\usepackage[utf8]{inputenc}
\usepackage[english]{babel}

\usepackage{amsthm}
\usepackage{amsmath}
\usepackage{amssymb}
\usepackage{mathtools}
\usepackage{bbm} % mathbb for numbers

% reset align numbering
\usepackage{etoolbox}
\AtBeginEnvironment{align}{\setcounter{equation}{0}}

\newtheorem{theorem}{Theorem}
\newtheorem{lemma}{Lemma}
\newtheorem{corollary}{Corollary}
\theoremstyle{definition}
\newtheorem{exercise}{Exercise}

\theoremstyle{definition}
\newtheorem{definition}{Definition}
\theoremstyle{remark}
\newtheorem{example}{Example}

\usepackage{color}
\usepackage{hyperref}
\hypersetup{
    colorlinks=true,
    linktoc=all,
    linkcolor=blue,
}

\DeclareMathOperator{\Var}{Var}
\DeclareMathOperator{\E}{\mathbb{E}}
\DeclareMathOperator{\I}{\mathbbm{1}}

\setlength\parindent{0pt}
\DeclareMathOperator*{\argmax}{arg\,max}
\DeclareMathOperator*{\argmin}{arg\,min}

%Header
\pagestyle{fancy}
\thispagestyle{empty}
\rhead{ \textit{ }} 

%% Title
\title{MOORE GRAPH WITH PARAMETERS (3250,57,0,1) DOES NOT EXIST
%%%% Cite as
%%%% Update your official citation here when published
\thanks{\textit{\underline{Citation}}: Translation from Russian:
\textbf{arXiv:2010.13443v2 [math.CO] 9 Nov 2020}}
}

\author{
    A.A.Makhnev \\
    Krasovskii Institute of Mathematics and Mechanics UB RAS, \\
    Ural Federal University \\
    \texttt{makhnev@imm.uran.ru} \\
}

\begin{document}

\maketitle

\begin{abstract}
If a regular graph of degree k and diameter d has v vertices then
\[ v \le 1 + k + k(k - 1) + \cdots + k(k - 1)^{d - 1} \]
Graphs with $v = 1 + k + k(k - 1) + · · · + k(k - 1)^{d-1}$ are called Moore graphs. Damerell proved that a Moore graph of degree $k \ge 3$ has diameter $2$. If $\Gamma$ is a Moore graph of diameter $2$, then $v = k^2 + 1$, $\Gamma$ is strongly regular with $\lambda = 0$ and $\mu = 1$, and one of the following statements holds: $k = 2$ and $\Gamma$ is the pentagon, $k = 3$ and $\Gamma$ is the Petersen graph, $k = 7$ and $\Gamma$ is the Hoffman-Singleton graph, or $k = 57$. The existence of a Moore graph of degree 57 was unknown.

Jurishich and Vidali have proved that the existence of a Moore graph of degree $k > 3$ is equivalent to the existence of a distance-regular graph with intersection array $\{k - 2, k - 3, 2; 1, 1, k - 3\}$ (in the case $k = 57$ we have a distance-regular graph with intersection array $\{55, 54, 2; 1, 1, 54\}$).

In this paper we prove that a distance-regular graph with intersection array $\{55, 54, 2; 1, 1, 54\}$ does not exist. As a corollary, we prove that a Moore graph of degree 57 does not exist.
K

\end{abstract}

\keywords{distance-regular graph \and Moore graph}.

\section{Intro}

We consider undirected graphs without loops and multi-edges. If $a,b$ are vertices of a graph $\Gamma$ then we denote by $d(a,b)$ the distance between $a$ and $b$. We denote by $\Gamma_i(a)$ a subgraph of the graph $\Gamma$ induced by a set of vertices with the distance $i$ from the vertex $a$. A subgraph $\Gamma_1(a)$ is called a {\it neighbourhood} of a vertex $a$ and denoted by $[a]$.

The definition of the {\it distance-regular graph} with the intersection array $\{b_0, ..., b_{d-1}; c_1, ..., c_d \}$ can be found in [1].

For a graph $\Gamma$ with the diameter $d$ and $i \in \{2,..,d\}$ let's denote by $\Gamma_i$ a graph defined on the same set of vertices where any two vertices $u,w$ are adjacent if the distance between $u$ and $w$ in the graph $\Gamma$ is $i$.

If a regular graph of degree $k$ and diameter $d$ has $k$ vertices then the following inequality holds:

\[
    v \le 1 + k + k(k-1) + \cdots + k(k-1)^{d-1}
\]

Graphs for which the non-strict inequality above turns into equality are called {\it Moore graphs} (1960). This definition belongs to Hoffman and Singleton who explored graphs with diameters of 2 and 3. The simplest example of Moore graph is a regular  $(2d +1)$-polygon.

Damerell [3] (see also [4]) has proven that a graph of degree $k \ge 3$ has diameter 2. In this case $ v = k^2 + 1$ the graph is strongly regular with $\lambda = 0$ and $ \mu = 1 $ and one of the following statements holds: $k = 2$ and the graph is the pentagon, $k = 3$ and the graph is the Petersen graph, $k = 7$ and the graph is the Hoffman-Singleton graph, or $k = 57$. The existence of a Moore graph of degree 57 was unknown.

Ashbacher [5] has proven that a Moore graph of degree 57 is not distance-transitive (the Moore graph of degree 57 sometimes called Ashbacher's graph). G. Higman ([6, Theorem 3.13]) has proven that a Moore graph of degree 57 is not a (vertex?)-symmetric graph. A.A. Makhnev and D.V. Paduchikh explored possible automorphisms of the Moore graph of degree 57.

Jurishich and Vidali [9] have noticed that the existence of a Moore graph of degree $k > 3$ is equivalent to the existence of a distance-regular graph with intersection array $\{k - 2, k - 3, 2; 1, 1, k - 3\}$ (in the case k = 57 we have a distance-regular graph with intersection array $\{55, 54, 2; 1, 1, 54\}$).

In this paper we explore properties of the graph with intersection array $\{55, 54, 2; 1, 1, 54\}$ and claim:

\begin{theorem}A distance-regular graph with intersection array $\{55, 54, 2; 1, 1, 54\}$ does not exist.
\label{th:1}
\end{theorem}

\begin{corollary}Strongly regular Moore graph with parameters $(3250, 57, 0, 1)$ does not exist.
\end{corollary}

This result puts an end to 60 years of research on Moore graphs. The main methods used to prove the results are: the triple intersection number method applied to a graph without the Q-polynomial condition; method of symmetrization of an array of triple intersection numbers (proposed by A.A. Makhnev, the analogue of tensor symmetrization).


\section{Auxiliary Results}

Our notation is conventional and can be found in [1]. In the proof of the theorem we use {\it triple intersection numbers} [11].

Let $\Gamma$ be a distance-regular graph with the diameter $d$. For vertices $u_1, u_2, u_3$ and non-negative integers $\Gamma$ and $r_1, r_2, r_3$ are non-greater then $d$ define as
$ \begin{Bmatrix}
    u_1 u_2 u_3 \\
    r_1 r_2 r_3
\end{Bmatrix}$ a set of vertices $w \in \Gamma$ such that $d(w, u_i) = r_i$.\footnote{Translator's note: I found the definition from [11] to be more precise
\[
\begin{Bmatrix}
    u_1 u_2 u_3 \\
    r_1 r_2 r_3
\end{Bmatrix} := \left\{ v \in \text{vertices}(\Gamma) \mid d(v, u_1) = r_1, d(v, u_2) = r_2, d(v, u_3) = r_3 \right\}
\]

}

Let's denote as
$\begin{bmatrix}
    u_1 u_2 u_3 \\
    r_1 r_2 r_3
\end{bmatrix}$ the number of vertices in
$\begin{Bmatrix}
    u_1 u_2 u_3 \\
    r_1 r_2 r_3
\end{Bmatrix}$. Numbers
$\begin{bmatrix}
    u_1 u_2 u_3 \\
    r_1 r_2 r_3
\end{bmatrix}$
we refer to as {\it triple intersection numbers}. For the fixed riplet of vertices $u_1,u_2,u_3$ instead of
$\begin{bmatrix}
    u_1 u_2 u_3 \\
    r_1 r_2 r_3
\end{bmatrix}$ we write
$[r_1 r_2 r_3]$.
Unfortunately there is no general way to compute $[r_1 r_2 r_3]$. But in [11] we propose a technique to compute some of $[r_1 r_2 r_3]$ numbers.
\\

Let $u,v,w$ be vertices of a graph $\Gamma$ and $W = d(u,v)$, $U = d(v,w)$, $V = d(u,w)$. There is only one vertex $x = u$ such that $d(x,u) = 0$. Therefore $[0jh]$ is equal to $0$ or $1$. Thus, $[0jh] = \delta_{jW}\delta_{hW}$, $[i0h] = \delta_{iW}\delta_{hU}$, $[ij0] = \delta_{iU}\delta_{jV}$.

Another set of equations we can get by fixing the distance between any two vertices from $\{u,v,w\}$ and by counting the number of vertices with any distance to the third one:


\[
    \sum_{l=1}^d [ljh] = p_{jh}^U - [0jh],
    \sum_{l=1}^d [ilh] = p_{ih}^V - [i0h],
    \sum_{l=1}^d [ijl] = p_{ij}^W - [ij0]
\label{eq:plus}
\tag{+}
\]

Wherein some of the triplets disappear. When $|i-j|>W$ or $|i+j|<W$ we have $p_{ij}^W = 0$. Therefore $[ijh]=0$ for all $h \in \{0, ..., d\}$.

Let $S_{ijh}(u,v,w) = \sum_{r,s,t=0}^d Q_{ri}Q_{sj}Q_{th} \begin{bmatrix}
    u v w \\
    r s t \end{bmatrix}$.
If {\it Krein } parameter $q_{ij}^h = 0$ then $S_{ijh}(u,v,w) = 0$.
\\

Let's fix vertices $u, v, w$ of a distance-regular graph $\Gamma$ with the diameter $3$ and define
$ \{ijh\} = \begin{Bmatrix} uvw \\ ijh \end{Bmatrix}$,
$ [ijh] = \begin{bmatrix} uvw \\ ijh \end{bmatrix}$,
$ [ijh]' = \begin{bmatrix} uwv \\ ihj \end{bmatrix}$,
$ [ijh]^* = \begin{bmatrix} vuw \\ jih \end{bmatrix}$,
$ [ijh]^{\sim} = \begin{bmatrix} wvu \\ hji \end{bmatrix}$
In cases where $d(u,v) = d(u,w) = d(v, w) = 2$ or $d(u,v) = d(u,w) = d(v, w) = 3$ computation of the parameters
$ [ijh]' = \begin{bmatrix} uwv \\ ihj \end{bmatrix}$,
$ [ijh]^* = \begin{bmatrix} vuw \\ jih \end{bmatrix}$,
$ [ijh]^{\sim} = \begin{bmatrix} wvu \\ hji \end{bmatrix}$
(symmetrization of the array of triple intersection numbers) can give new ratios to prove the non-existence of the graph.

\section{Properties of $\Gamma_3(u)$ graph}

In this section the graph $\Gamma$ is a distance-regular graph with intersection array $\{55, 54, 2; 1, 1, 54\}$. Then $\Gamma$ has the spectrum $55^1, 7^{1617}, -1^{110}, -8^{1408}, 1 + 55 + 2970 + 110 = 3136$ vertices and dual eigenvalue matrix:

\renewcommand{\arraystretch}{2.2}
\[
\begin{pmatrix}
    1 & 1617 & 110 & 1408 \\
    1 & \dfrac{1029}{5} & -2 & -\dfrac{1024}{5} \\
    1 & -\dfrac{49}{15} & -2 & \dfrac{64}{15} \\
    1 & \dfrac{147}{5} & 54 & -\dfrac{128}{5}
\end{pmatrix}
\]
\renewcommand{\arraystretch}{1.0}
\\

Then the graph $\Gamma_3$ is $56 \times 56$ lattice graph and the neighborhood of the vertex in $\Gamma_3$ is a union of two isolated $55$-cliques.

\begin{lemma}The intersection numbers of the graph $\Gamma$ are:
    \begin{enumerate}
	\item $p_{11}^1 = 0, p_{12}^1 = 54, p_{22}^1 = 2808, p_{23}^1 = 108, p_{33}^1 = 2$
	\item $p_{11}^2 = 1, p_{12}^2 = 52, p_{13}^2 = 2, p_{22}^2 = 2811, p_{23}^2 = 106, p_{33}^2 = 2$
	\item $p_{12}^3 = 54, p_{13}^3 = 1, p_{22}^3 = 2862, p_{23}^3 = 54, p_{33}^3 = 54$
    \end{enumerate}
\end{lemma}

\begin{proof} Direct computation according to [1, Lemma 4.1.7].
    Let's fix the vertices $u,v,w$ of the graph $\Gamma$ and define
$ \{ijh\} = \begin{Bmatrix} uvw \\ ijh \end{Bmatrix}$,
$ [ijh] = \begin{bmatrix} uvw \\ ijh \end{bmatrix}$.

Let $\Delta = \Gamma_2(u)$ and $\Lambda = \Delta_2$. Then $\Lambda$ is a regular graph of degree $p^2_{22} = 2811$ defined on $k_2 = 2970$ vertices.
\end{proof}

\begin{lemma} Let $d(u,v) = 2, d(u,w) = d(v,w) = 1$. Then triple intersection numbers are:
    \begin{enumerate}
	\item $[122] = 52, [132] = 2$
	\item $[212] = 52, [221] = 53, [222] = r_1 + 2650, [223] = -r_1 + 108, [232] = -r_1 + 106, [233] = r_1$
	\item $[312] = 2, [322] = -r_1 + 106, [323] = [332] = r_1, [333] = -r_1 + 2$
    \end{enumerate}
    where $r_1 \in \{0,1,2\}$
\label{lem:2}
\end{lemma}

\begin{proof} Simplification of the set of equations \ref{eq:plus}.
\end{proof}

\begin{lemma} Let $d(u,v) = d(u, w) = 2, d(v,w) = 3$. Then triple intersection numbers are:
    \begin{enumerate}
	\item $[112] = [121] = -r_6 + 1, [113] = [131] = r_6, [122] = r_7, [123] = [132] = r_6 - r_7 + 51, [133] = -2r_6 + r_7 - 49$
	\item $[212] = [221] = 52, [213] = [231] = 0, [222] = r_5 + r_6 + 2756 - r_7 = r_9, [223] = [232] = -2r_6 + r_7 + 3, [233] = 2r_6 - r_7 + 102$
	\item $[312] = [321] = r_6 + 1, [313] = [331] = 1 - r_6, [322] = 5 - 2 r_6, [323] = [332] = r_6, [333] = 1$
    \end{enumerate}
    where $r_6 \in \{0,1\}$, $r_7 \in \{49, ..., 52 \}$. Then one of the followings is true:
    \[[222] = 2706 \text{ and } [222] = 2r_6 + 50 = r_7 \]
    or
    \[[222] = 2707 \text{ and } [222] = 2r_6 + 51 = r_7 \text{ and } r_6 = 0 \]
\end{lemma}
\begin{proof} Using computer aided simplification of the set of equations \ref{eq:plus} we have

$[112] = -r_{6} +1, [113] = r_{6} , [121] = -r_{4} -r_{8} +54, [122] = r_{7} , [123] = r_{4} -r_{7} +r_{8} -2, [131] = r_{4} +r_{8} -53, [132] = r_{6} - r_{7} + 51, [133] = -r_{4} - r_{6} + r_{7} - r_{8} + 4;$ \\
$[212] = r_{5} + r_{6} - r_{7} - r_{9} + 2808, [213] = -r_{5} - r_{6} + r_{7} + r_{9} - 2756, [221] = r_{4} , [222] = r_{9} , [223] = -r_{4} - r_{9} + 2811, [231] = -r_{4} + 52, [232] = -r_{5} - r_{6} + r_{7} + 3, [233] = r_{4} + r_{5} + r_{6} - r_{7} + 50;$ \\
$[312] = -r_{5} + r_{7} + r_{9} - 2755, [313] = r_{5} - r_{7} - r_{9} + 2757, [321] = r_{8} , [322] = -r_{7} - r_{9} + 2861, [323] = r_{7} - r_{8} + r_{9} - 2755, [331] = -r_{8} + 2, [332] = r_{5} , [333] = -r_{5} + r_{8}$, \\

where $r_4 \in \{51, 52\}, r_5 \in \{0, 1, 2\}, r_6 \in  \{0, 1\}, r_7 \in \{49, ..., 52\}, r_8 \in \{0, 1, 2\}, r_9 \in \{2705, ..., 2710\}$.
Thus $2705 \le [222] = r_9 \le 2710$. \\

In the $56 \times 56$ lattice $\Gamma_3$ a subgraph $\Gamma_3(u) \cap \Gamma_3(v)$ is 2-co-clique which has intersection with $\Gamma_3(w)$ exactly in one vertex. Therefore $[333] = -r_5 + r_8 = 1$ and $r_5 \in \{0, 1\}$.

Same, $\Gamma_3(v) \cap \Gamma_3(w)$ is 54-clique which has intersection just with $\Gamma_3(u)$ in no more then one vertex. Therefore $[133] = - r_4 - r_6 + r_7 - r_5 + 3 \le 1$. \\

Symmetrization. For the vertex triplet  $(u, v, w)$ we have $[122] = r'_7 = r'_7 , -r_6 + 1 = [112] = [121]' = -r'_4 - r'_5 + 53$ and $r_4 + r_5 = r'_6 + 52$. Then, $[222] = r_9 = r'_9$, $r_6 = [323] = [332]' = r'_5$ and  $r_4 = 52$. Therefore $[133] = -49 - r_6 + r_7 - r_5 \le 1$ and $ r_7 \le r_6 + r_5 + 50$.

Then we have $[313] = r_5 - r_7 - r_9 + 2757$, therefore
$r_7 + r_9 \le r_5 + 2757$ and $[213] = -r_5 - r_6 + r_7 + r_9 - 2756 \le 1 - r_6$.

Assume $[213] = 1$. Then $r_6 = 0, [233] = r_5 -r_7 + 102$ and $r_5 = r'_5 = 0$, thus $r_7 +r_9 = 2757$.

Because $[133] = r_7 - r_8 - 48 = r'_7 - r'_8 - 48$ then $r_8 = r'_8$. Further, $1 = [112] = [121]' = r'_8$ and $r_8 = 1$. Same,
$[233] = r_4 - r_7 + 50$, therefore, $r_4 = r'_4$. We have a contradiction because $51 = [212] = [221]' = r'_4 = 52$. Therefore $[213] = 0$ and $r_7 + r_9 = r_5 + r_6 + 2756$. If $r_6 = 0$, then $[233] = r_5 - r_7 + 102$ and $r_5 = r'_5 = 1$.\\

In case if $r_6 = 1$, then $[233] = r_5 - r_7 + 103$ and  $r_5 = r'_5 = 1$. Thus in either cases we have $r_5 = r_6$ and we have symmetrized array:

 $[112] = [121] = -r_6 + 1, [113] = [131] = r_6 , [122] = r_7 , [123] = [132] = r_6 - r_7 + 51, [133] = -2r_6 + r_7 - 49;$

$[212] = [221] = 52, [213] = [231] = 0, [222] = r_5 + r_6 + 2756 - r_7 = r_9 , [223] = [232] = -2r_6 + r_7 + 3, [233] = 2r_6 - r_7 + 102;$

$[312] = [321] = r_6 + 1, [313] = [331] = 1 - r_6, [322] = 5 - 2r_6 , [323] = [332] = r_6 , [333] = 1$

where $r_6 \in \{0, 1\}, r_7 \in \{49, ..., 52\}$. \\

If $[222] \ge 2708$ then $r_7 \le 48 + 2r_6$ and the equation $[133] = -2r_6 + r_7 - 49$ gives a contradiction.
If $[222] = 2705$, then $r_7 = 51 + 2r_6$, we have contradiction with the fact $[133] = -2r_6 + r_7 - 49 \le 1$. So, either
$[222] = 2706$ and $[222] = r_5 + r_6 + 50 = r_7$ or $[222] = 2707, r_5 + r_6 + 51 = r_7$ and $r_5 + r_6 = 0$.

\end{proof}

\begin{lemma} Let $d(u, v) = d(u, w) = 2$, $d(v, w) = 1$. Then triple intersection numbers are:
    \begin{enumerate}
	\item $[122] = 51, [123] = [132] = 0, [133] = 2$;
	\item $[212] = [221] = 51, [222] = 2654, [223] = [232] = 106, [233] = 0$;
	\item $[312] = [321] = 2, [322] = 102, [323] = [332] = 2, [333] = 0$.
    \end{enumerate}
Further, $[222] = 2654$.
\label{lem:4}
\end{lemma}

\begin{proof} Using simplification of the formula set \ref{eq:plus} we get:

$[122] = r_3 , [123] = [132] = -r_3 + 51, [133] = r_3 - 49$;

$[212] = [221] = 51, [222] = -r_2 - r_3 + 2705, [223] = [232] = r_2 + r_3 + 55, [233] = -r_2 - r_3 + 51$;

$[312] = [321] = 2, [322] = r_2 + 102, [323] = [332] = -r_2 + 2, [333] = r_2$

where $r_2 \in \{0, 1, 2\}, r_3 \in \{49, 50, 51\}$. \\

In the $56 \times 56$-lattice $\Gamma_3$ a subgraph $\Gamma_3(u) \cap \Gamma_3(w)$ is 2-co-clique. It has no intersection with $\Gamma_3(v)$. Therefore $[333] = r_2 = 0$. \\

Let's compute a number $d$ of pairs of vertices $(y,z)$ with the distance $3$ in the graph $\Lambda$ where
 $y \in \begin{Bmatrix}uv \\ 2 1 \end{Bmatrix}$,
 $z \in \begin{Bmatrix}uv \\ 2 3 \end{Bmatrix}$.

On the one hand we have $[213] = 0$ and $d = 0$ by the Lemma \ref{lem:2}. On the other hand we have $ d = 52[233] = 0 $, therefore $[233] = 0$ and $r_3 = 51$. Thereby we have the equations from the lemma statement.
\end{proof}

\begin{lemma} Let $d(u, v) = d(u, w) = (v, w) = 2$. Then triple intersection numbers are: \\
$[111] = -r_{16} - r_{11} + 1, [112] = r_{11} , [113] = r_{16} , [121] = r_{14} , [122] = -r_{12} - r_{14} + 52, [123] = r_{12}, [131] = -r_{14} + r_{16} + r_{11} , [132] = r_{12} + r_{14} - r_{11} , [133] = -r_{12} - r_{16} + 2$;

$[211] = r_{15} + r_{16} - r_{17} + 50, [212] = r_{17} , [213] = -r_{15} - r_{16} + 2, [221] = -r_{15} - r_{16} + r_{17} - r_{10} + 2, [222] = r_{12} + r_{16} - r_{17} + r_{10} + 2704, [223] = -r_{12} + r_{15} + 104, [231] = r_{10} , [232] = -r_{12} - r_{16} - r_{10} + 106, [233] = r_{12} + r_{16}$;

$[311] = -r_{15} +r_{17} +r_{11} -50, [312] = -r_{17} -r_{11} +52, [313] = r_{15} , [321] = -r_{14} +r_{15} +r_{16} -r_{17} +r_{10} +50, [322] = r_{14} -r_{16} +r_{17} -r_{10} +54, [323] = -r_{15} +2, [331] = r_{14} -r_{16} -r_{10} -r_{11} +2, [332] = -r_{14} +r_{16} +r_{10} +r_{11}, [333] = 0$ \\

where $r_{10}, r_{12}, r_{15} \in \{0, 1, 2\}, r_{11} , r_{14} , r_{16} \in \{0, 1\}, r_{17} \in \{49, ..., 52\}$.

Further, $2652 \le [222] = r_{12} + r_{16} - r_{17} + r_{10} + 2704 \le 2659$.

\end{lemma}

\begin{proof} With computer-aided simplification of \ref{eq:plus} we have:

$[111] = -r_{16} - r_{11} + 1, [112] = r_{11} , [113] = r_{16} , [121] = r_{14} , [122] = -r_{12} - r_{14} + 52, [123] = r_{12} , [131] = -r_{14} + r_{16} + r_{11} , [132] = r_{12} + r_{14} - r_{11} , [133] = -r_{12} - r_{16} + 2$;

$[211] = r_{15} + r_{16} - r_{17} + 50, [212] = r_{17} , [213] = -r_{15} - r_{16} + 2, [221] = -r_{15} - r_{16} + r_{17} - r_{10} + 2, [222] = r_{12} -r_{13} +r_{16} -r_{17} +r_{10} +2704, [223] = -r_{12} +r_{13} +r_{15} +104, [231] = r_{10} , [232] = -r_{12} +r_{13} -r_{16} -r_{10} +106, [233] = r_{12} - r_{13} + r_{16}$;

$[311] = -r_{15} +r_{17} +r_{11} -50, [312] = -r_{17} -r_{11} +52, [313] = r_{15} , [321] = -r_{14} +r_{15} +r_{16} -r_{17} +r_{10} +50, [322] = r_{13} + r_{14} - r_{16} + r_{17} - r_{10} + 54, [323] = -r_{13} - r_{15} + 2, [331] = r_{14} - r_{16} - r_{10} - r_{11} + 2, [332] = -r_{13} - r_{14} + r_{16} + r_{10} + r_{11} , [333] = r_{13}$

where $r_{10} , r_{12} , r_{13} , r_{15} \in \{0, 1, 2\}, r_{11} , r_{14} , r_{16} \in \{0, 1\}, r_{17} \in \{49, ..., 52\}$.

In the $56 \times 56$-lattice $\Gamma_3$ a subgraph $\Gamma_3(u) \cap \Gamma_3(w)$ is 2-co-clique. It has no intersection with $\Gamma_3(v)$. Therefore $[333] = r_{13} = 0$. Thereby we have the required equations.

Let's note that $r_{12} + r_{16} \le 2$, therefore $2652 \le [222] = r_{12} + r_{16} - r_{17} + r_{10} + 2704 \le 2659$. Lemma is proved.
\end{proof}
For number $e$ of edges between $\Lambda(v)$ and $\Lambda_2(v)$ in a graph $\Lambda$ the following inequalities hold: $424844 = 52 \cdot 2654 + 106 \cdot 2706 \le e \le 52 \cdot 2654 + 106 \cdot 2707 = 424950$. Therefore $ 151.136 \le 2810 - \lambda \le 151.174$ and $2658.826 \le \lambda \le 2658.864$, where $\lambda$ is a mean value of the parameter $\lambda(\Lambda)$.

Let's note that the mean value $\lambda(\Lambda)$ is very close to the upper bound $2659$.

\section{Proof of the theorem}

Let $\Gamma$ be a distance-regular graph with an intersection array $\{55, 54, 2; 1, 1, 54\}$. Here we will prove the Theorem \ref{th:1}.
Let's fix $(u,v,w)$ vertices of the graph $\Gamma$ and define
$\{ijh\} = \begin{Bmatrix}uvw \\ ijh \end{Bmatrix}$,
$[ijh] = \begin{bmatrix}uvw \\ ijh \end{bmatrix}$.
Let $\Delta = \Gamma_2(u)$ and $\Lambda = \Delta_2$. Then $\Lambda$ is a regular graph with the degree $p^2_{22} = 2811$ defined on $k_2 = 2970$ vertices.

\begin{lemma} Let $d(u, v) = d(u, w) = (v, w) = 2$. Then the following statements hold:
\begin{enumerate}
    \item $r_{11} = r'_{14}, r'_{10} = [213] = r^*_{12}, r_{15} + r_{16} + r'_{10} = 2$ and  $-r_{17} + 52 = r^*_{12} + r^*_{14}$
    \item if $r'_{10} = 2$ then $r_{10} = 2$ and one of the following cases is true:

either $r_{11} = 0$ and \\
$[111] = 1, [112] = [113] = [121] = [131] = 0, [122] = 50, [123] = [132] = 2, [133] = 0;$ \\
$[211] = 0, [212] = r_{17} = 50, [213] = [231] = 2, [221] = 50, [222] = 2658, [223] = [232] = 102, [233] = 2;$ \\
$[311] = 0, [312] = [321] = 2, [313] = 0, [322] = 102, [323] = [332] = 2, [331] = [333] = 0$,\\

or $r_{11} = 1$ and \\
$[111] = 0, [112] = [121] = 1, [113] = [131] = 0, [122] = 49, [123] = [132] = 2, [133] = 0;$ \\
$[211] = 1, [212] = [221] = r_{17} = 49, [213] = [231] = 2, [222] = 2659, [223] = [232] = 102, [233] = 2;$ \\
$[311] = 0, [312] = [321] = 2, [313] = [331] = 0, [322] = 102, [323] = [332] = 2, [333] = 0.$

\end{enumerate}

\end{lemma}

\begin{proof} Symmetrization. From $[111] = -r_{16} - r_{11} + 1$ we have $r_{16} + r_{11} = r'_{16} + r'_{11} = r^*_{16} + r^*_{11} = r^\sim_{16} + r^\sim_{11}$. Then
 $[112] = r_{11} = r*_{11} , [113] = r_{16} = r^*_{16} , [121] = r_{14} = r^\sim_{14} , [212] = r_{17} = r^\sim_{17}, [313] = r_{15} = r^\sim_{15}, r_{11} = [112] = [121]' = r'_{14}, r'_{10} = [213] = r^*_{12}$. \\

From $[122] = -r_{12} - r_{14} + 52$ follows $r_{12} + r_{14} = r'_{12} + r'_{14}, [233] = r_{12} + r_{16} = r'_{12} + r'_{16}, r_{10} = [231] = [213]' = -r'_{15} - r'_{16} + 2$ and $r'_{10} + r_{15} + r_{16} = 2$. \\

Similarly, $-r_{17} - r_{11} + 52 = [312] = [132]^* = r^*_{12} + r^*_{14} - r^*_{11}$ and  $-r_{17} + 52 = r^*_{12} + r^*_{14}$. The statement (1) is proved.\\

Let's assume $ r'_{10} = 2$. Then $r^*_{12} = 2, r_{15} + r_{16} = 0$ and $r_{17} + r^*_{14} = 50$. Because $[131] = -r_{14} + r_{11}$ then in the case of $r_{11} = 0$ we have $r_{14} = 0, [132] = r_{12} + r_{14} - r_{11} = r_{12} = r'_{12}, [211] = -r_{17} + 50, [311] = r_{17} - 50$. Now $-r_{12} - r_{10} + 106 = [232] = [223]' = -r'_{12} + 104 = -r_{12} + 104$, therefore $r_{10} = 2, [212] = [221] = r_{17} = 50, r_{17} = [122] = -r_{12} + 52, r_{12} = 2$ and \\

$[111] = 1, [112] = [113] = [121] = [131] = 0, [122] = 50, [123] = [132] = 2, [133] = 0;$
$[211] = 0, [212] = r_{17} = 50, [213] = [231] = 2, [221] = 50, [222] = 2658, [223] = [232] = 102, [233] = 2;$
$[311] = 0, [312] = [321] = 2, [313] = 0, [322] = 102, [323] = [332] = 2, [331] = [333] = 0.$ \\

In case where $r_{11} = 1$ we have $[233] = r_{12} = r'_{12}$, thereby $r_{12} = [123] = [132] = [132] = r_{12} + r_{14} - 1, r_{14} = 1, -r_{17} + 51 = [312] = [321]' = -r'_{17} + r'_{10} + 49, r_{17} = r'_{17}, r_{10} = 2$ and $-r_{17} +51 = [312] = [132]^* = r^*_{12}$. \\

We have $[222] = r_{12} + 2657$, thereby $r_{12} = r^*_{12}, r_{12} + r_{17} = 51$ and \\
$[111] = 0, [112] = [121] = 1, [113] = [131] = 0, [122] = 49, [123] = [132] = 2, [133] = 0;$ \\
$[211] = 1, [212] = [221] = r_{17} = 49, [213] = [231] = 2, [222] = 2659, [223] = [232] = 102, [233] = 2;$ \\
$[311] = 0, [312] = [321] = 2, [313] = [331] = 0, [322] = 102, [323] = [332] = 2, [333] = 0.$ \\

Now $[212] = [221] = r_{17} = r^*_{17}, r_{17} = [122] = -r_{12} + 51$ and $r_{12} = 2$. Lemma is proved.
\end{proof}


The following idea from Vidali can be used to simplify the proof of the theorem. Let $d(u, v) = 2$. Then $[u]$ has one element in each row and each column of the $55 \times 55$-lattice $\Gamma_3 - (\{u\} \cap \Gamma_3(u))$
Let $\{x, y\} = [u] \cap \Gamma_3 (v)$. Then there is only one vertex in the subgraph $\Gamma_3(x) \cap \Gamma_3(y) - \{v\}$. We have a contradiction with the fact $[133] = 2]$ (see Lemma \ref{lem:4}) for any three vertices $(u,v,w)$ with $d(u, v) = d(u, w) = 2, d(v, w) = 1$.

Theorem \ref{th:1} is proved.

\begin{thebibliography}{}

\bibitem[1]{1} Brouwer A. E., Cohen A. M., Neumaier A. Distance-Regular Graphs. Berlin; Heidelberg; New York: Springer-Verlag, 1989.
\bibitem[2]{2} Hoffman Alan J., Singleton Robert R. Moore graphs with diameter 2 and 3 // IBM Journal of Research and Development. 1960. V. 5, N 4. P. 497–504.
\bibitem[3]{3} Damerell R.M. On Moore graphs // Math. Proc. Cambr. Phil. Soc., Vol. 74 (1973). P. 227–236.
\bibitem[4]{4} Bannai E., Ito T. On finite Moore graphs // Journal of the Faculty of Science, the University of Tokyo. Sect. 1 A, Mathematics. 1973, V. 20. P. 191-208.
\bibitem[5]{5} Ashbacher M. The nonexistence of rank 3 permutation group of degree 3250 and subdegree 57 // J.Algebra, Vol. 19, no. 3 (1971). P. 538-540.
\bibitem[6]{6} Cameron P.J. Permutation Groups. London Math. Soc. Student Texts N 45: Cambridge Univ. Press, 1999
\bibitem[7]{7} Makhnev A.A., Paduchikh D.V. On automorphisms of Aschbacher graph // Algebra i Logika, Vol. 40, no. 2 (2001). P. 125-134.
\bibitem[8]{8} Makhnev A.A., Paduchikh D.V. On automorphisms group of Aschbacher graph // Doklady RAN, Vol. 426, no. 3 (2009). P. 310-313.
\bibitem[9]{9} Jurishich A., Vidali J. The Sylvester graph and Moore graphs // Europ. J. Comb., Vol. 80 (2019). P. 184-193.
\bibitem[10]{10} Makhnev A.A., Paduchikh D.V. The most Moore graph and distance-regular graph with intersection array $\{55, 54, 2; 1, 1, 54\}$ // Algebra i Logika, Vol. 59, no. 4 (2020).
\bibitem[11]{11} Jurishich A., Vidali J. Extremal 1-codes in distance-regular graphs of diameter 3 // Des. Codes Cryptogr., Vol. 65 (2012). P. 29–47.
\end{thebibliography}

\section*{About Authors}
MAKHNEV Alexander Alexeevich \\
Russia, Ekaterinburg, S.Kovalevskoy st, 16, 620990, \\
N.N. Krasovskii Institute of Mathematics and Mechanics \\
Russia, Ekaterinburg, Mira st, 19, 620002, \\
Ural Federal University \\
makhnev@imm.uran.ru

\section*{About Translators}
Selivanov Konstantin, konstantin.selivanov@gmail.com

\end{document}
